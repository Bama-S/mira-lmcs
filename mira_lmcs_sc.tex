\documentclass[a4paper,11pt]{lmcs}
\usepackage[utf8]{inputenc}
\usepackage{amsmath}
\usepackage{amssymb}
\usepackage{amsthm}
\usepackage{mathrsfs}
\usepackage{natbib}
\newcommand{\bigseq}{\mathop{\raisebox{0.3ex}{\scalebox{1.5}{$\prod$}}}}
\newcommand{\paralleli}{\mathrel{\|_i}}




% Specify the name of your bibliography file

%\newtheorem{proposition}[theorem]{Proposition}%
%\newtheorem{proposition}{Proposition}% to get separate numbers for theorem and proposition etc.

%\theoremstyle{thmstyletwo}%
%\newtheorem{example}{Example}%
%\newtheorem{remark}{Remark}%

%\theoremstyle{thmstylethree}%
\newtheorem{theorem}{Theorem}
\newtheorem{definition}{Definition}%

\newcommand{\mimamsa}{M\={i}m\={a}\.ms\={a}}
%opening
\title{A M\={i}m\={a}\.{m}s\={a} Inspired Framework for Instruction Sequencing in AI Agents}
\author{Bama Srinivasan}
%\lmcsorcid{0002-8698-1836}
\address{Department of Information Science and Technology, CEG Campus, Anna University, India}
\email{bama@auist.net}
\begin{document}

\maketitle

\begin{abstract}
This paper presents a formal framework for sequencing instructions in AI agents, inspired by the Indian philosophical system of \mimamsa. The framework formalizes sequencing mechanisms through action–object pairs in three distinct ways: direct assertion (Śrutikrama) for temporal precedence, purpose-driven sequencing (Arthakrama) for functional dependencies, and iterative procedures (Pravṛttikrama) for distinguishing between parallel and sequential execution in repetitive tasks. It introduces the syntax and semantics of an action–object imperative logic, extending the MIRA formalism \citep{mira} with explicit deduction rules for sequencing. The correctness of instruction sequencing is established through a validated theorem, which is based on object dependencies across successive instructions. This is further supported by proofs of soundness and completeness. This formal verification enables reliable instruction sequencing, impacting AI applications across areas like task planning and robotics by addressing temporal reasoning and dependency modeling

\end{abstract}

\section{Introduction}
Assume a robot is given two different tasks. The first one is preparing a tomato noodle stir-fry, and the second one, painting 10 fence panels. The cooking task requires steps such as "pick noodles," "cook noodles," "chop tomato," and "fry and combine," where chopping precedes frying. The painting task involves at least three actions such as applying primer, first coat, and second coat to each panel, which can be performed either panel-by-panel or by completing each stage across all panels sequentially. These examples illustrate the challenges of determining action order, object usage, and optimal sequencing to maximize efficiency, thereby motivating the need for a robust sequencing framework.

To address such challenges, numerous studies have explored solutions across different domains, such as hierarchical planning, temporal logic, Belief-Desire-Intention (BDI) models, Natural Language Processing, and Large Language Models. While these methods have advanced the research in terms of reasoning about time, intention, and structure, they often lack a principled framework for interpreting and sequencing imperatives.

This paper aims to address the aforementioned challenges by formalizing various instruction sequencing methods inspired by the Indian philosophical system of \mimamsa. This philosophy provides a systematic and logical method for resolving ambiguities in instructions, thereby helping to determine their proper sequence. The sequencing methods include: direct assertion sequencing (\textit{'Srutikrama}), purpose-driven sequencing (\textit{Arthakrama}), and step-by-step parallel sequencing (\textit{Pravṛttikrama}). Extending on the framework of MIRA \citep{mira} and temporal reasoning approaches \citep{llmmira}, instructions are treated as action–object pairs in this paper, which serve as the fundamental unit for sequencing in these three methods.

The contributions of this paper are as follows:
\begin{itemize}
\item Modeling action–object pairs to capture dependency relations.
\item Formalization of sequencing methods: direct assertion, purpose-driven, and step-by-step parallel sequencing versus traditional sequential completion.
\item Introduction of object consistency and functional dependency concepts, along with formal proofs validating sequence correctness.
\item Deduction rules for sequence validity, supported by soundness and completeness proofs.
\end{itemize}

The rest of this paper is organized as follows. Section \ref{sec:related} discusses related work. Section \ref{sec:syn_sem} describes the syntax and semantics of the action–object imperative logic. Section \ref{sec:sequencing} presents the formal representations of the three sequencing methods. Section \ref{sec:dependency} explains the object and functional dependencies, introduces the validity theorem with its proof, and outlines additional deduction rules for sequencing. Section \ref{sec:sound_complete} establishes the soundness and completeness of the proposed sequencing methods. Finally, Section \ref{sec:conclude} summarizes the findings and outlines directions for future work.

\section{Related Work}
\label{sec:related}
This section reviews key developments in classical and temporal planning, hierarchical methods, reasoning about actions and change, plan recognition, dependency modeling, BDI reasoning, normative systems, imperative logic, and LLM-based task planning. Based on this review, the proposed framework is positioned as a logical approach that addresses instruction sequencing.

\subsection*{Classical Planning and Temporal Logic}
Traditional planning systems, such as STRIPS, consider actions as transitions among states \citep{strips}. Each action is described using classical logic formulas that define its preconditions and effects. A search process is employed to navigate from the initial state to the goal. The resulting path represents a sequence of actions that together form a plan.

Although this method is robust, there are some disadvantages:
\begin{itemize}
 \item Simultaneous actions cannot be represented
 \item Representation of composite actions is difficult
 \item External event representation is not possible
\end{itemize}
To some extent, these challenges have been mitigated by incorporating temporal logic into planning frameworks \citep{allen}. Temporal logic deals with reasoning about time using the operators $\square$ (\textit{always}) and $\Diamond$ (\textit{eventually}), $\bigcirc$ (\textit{next}) and $U$ (\textit{Until}) \citep{huth}. In this approach, events are treated as objects, and necessary conditions—such as preconditions—are expressed over temporal intervals. This allows for the construction of non-linear plans and provides the flexibility to represent concurrent actions. However, this method cannot be used to break the complex task into simple ones. To overcome this limitation, a hierarchical planning structure is adopted, incorporating temporal reasoning across subtasks \citep{fei}.
Extending the work of Allen, Henry A. Kautz and Peter B. Ladkin has provided a temporal reasoning system that handles dates, duration and disjointness \citep{kautz}.


\subsection*{Reasoning About Actions and Change}
Research on action and change has been a continuing focus within the field of knowledge representation and reasoning about actions. As part of non-monotonic reasoning, Lifschitz applied the concept of circumscription to formalize the effects of actions in terms of causality and preconditions \citep{lifschitz1987formal}. This formalization provides a logical framework for addressing both the frame problem and the qualification problem. Subsequent to these, there has been several works on action languages A, B, C that include causal laws, preconditions and indirect effects in logic programming framework \citep{lifschitz1999action,gelfond1998representing,gelfond1998languages}. Their integration with answer set programming (ASP) enabled agents to reason about change using declarative models.

These theories offer a formal basis for representing sequencing and dependency reasoning, ideas that are further extended in this proposed \mimamsa-inspired framework, where textual imperatives are interpreted as causal action objects derived from a hermeneutic framework.

\subsection*{Hierarchical Planning}
In hierarchial planning, the top-level tasks are split into primitive actions. These are recursively completed to achieve the end task \citep{hplan2}. The most common framework for this approach is Hierarchical Task Network (HTN), which employs rule-based decomposition methods to break down compound tasks into smaller task networks. This allows the explicit representation of procedural knowledge and domain-specific constraints \citep{htn1}.

Hierarchical planning has proven effective in intelligent user assistance scenarios - For instance, guiding the assembly of a complex home theater system with components such as a Blu-ray player, amplifier, and television \citep{hplan1}. In such a case, the overarching goal might begin as a compound task like \textit{connect(Blu-ray, TV)}. This abstract task is then refined through decomposition rules that specify how the devices should be connected—either directly using primitive actions or indirectly through an intermediate device. By embedding constraints such as signal flow within the hierarchy, the planner efficiently narrows the search space and produces valid, step-by-step instructional guidance.
\subsection*{Plan Recognition and Goal Inference}
In the study of activity, plan, and goal recognition, the authors have examined these as a unified problem that reflects the natural way humans perceive such processes \citep{goal_act_plan}. From a computational perspective, four main approaches have been explored: logic-based methods, which are structured and rigid; classical machine learning techniques, which employ statistical models to manage uncertainty and noisy data; deep learning approaches, which automatically extract features from sensor data and are particularly effective for activity recognition; and brain-inspired models, which address ambiguity in real-world contexts. The review highlights the need for a unified framework that integrates the strengths of all these approaches.

In the paper “Goal and Plan Recognition Design for Plan Libraries,” the authors extend the traditional Goal Recognition Design (GRD), originally defined for STRIPS domains, to one that incorporates plan libraries, enabling the representation of hierarchical structures and complex agent behaviors \citep{plan_library}. They define two related problems: Goal Recognition Design for Plan Libraries (GRD-PL), which focuses on identifying what goal an agent is pursuing, and Plan Recognition Design (PRD), which focuses on how the goal is achieved. Two key metrics are introduced here, namely Worst-Case Distinctiveness (\textit{wcd}), measuring the maximum number of observations needed to unambiguously determine the goal, and Worst-Case Plan Distinctiveness (\textit{wcpd}), for determining the specific plan. The study demonstrates that \textit{wcd} serves as a theoretical lower bound for \textit{wcpd}, offering domain designers practical tools to quantify and minimize ambiguity, thereby enhancing the accuracy and efficiency of real-time plan and goal recognition systems.

\subsection*{Dependency and Semantic Relations}
Several studies have focused on identifying dependency relations among words in natural language sentences. In Semantic Role Labeling (SRL), the verb is treated as the root or predicate, and the relationships of other words to this predicate are analyzed \citep{srl}. For instance, in the sentence \textit{“Chrysler plans new investment in Latin America”}, the verb \textit{`plans'} serves as the predicate, semantically linked to the agent (\textit{Chrysler}) and the patient (\textit{investment}).

In dependency parsing, each word in a sentence is connected to another word, except for the main verb, which serves as the root. This word functions as the root node. For example, in the sentence “A man sleeps”, \textit{`sleeps'} is the root, \textit{`man'} depends on \textit{`sleeps'}, and \textit{`a'} depends on \textit{`man'} \citep{srl_survey}.  These word-to-word dependencies help in identifying the semantic relationships within a text.
\subsection*{BDI Model and Practical Reasoning}
Analogous to how humans select actions based on their goals, the Belief–Desire–Intention (BDI) model includes an intelligent agent with goal-oriented reasoning by representing goals as desires and actionable commitments as intentions, guided by the agent’s current beliefs \citep{bdi}, \citep{bdi_reasoning}. For instance, if a robot’s goal is to deliver a package, it might form a plan and commit to executing it—such as “take hallway A to room 10.” However, if it discovers that the hallway is blocked (a new belief), it will revise its intention and replan accordingly. Thus, the framework enables the agent to continuously update its knowledge while remaining aligned with its goals.


\subsection*{Normative Systems and Philosophical Foundations}
Normative logic in both Western and Indian philosophical traditions shares a common view on how rules and obligations guide rational action \citep{sanyal}. Deontic logic formalizes such norms through concepts of obligations and prohibitions, offering a useful framework for reasoning in multi-agent systems. Similarly, the Indian philosophical system of \mimamsa presents a sophisticated hermeneutical framework for interpreting texts, providing well-developed principles to distinguish between primary and auxiliary duties as well as exceptions \citep{davis}.

\subsection*{Imperative Logic and Action Representation}
Imperatives differ from declarative statements in that they cannot be evaluated as \textit{True} or \textit{False}, which limits the applicability of classical logic in reasoning about them. To address this, several studies have explored translating imperatives into propositions and then applying classical logic rules. However, such approaches often encounter paradoxes, as highlighted by Ross, revealing fundamental limitations in representing imperatives within standard logical frameworks. A detailed survey of various perspectives on imperative logic, particularly in the context of action representation in AI, has been presented in the survey of imperatives and action representation formalisms \citep{survey}. Inspired by the Indian philosophical system of \mimamsa, a three-valued formalism has been proposed to handle imperatives, where each instruction is evaluated as \textit{Satisfaction} ($S$), \textit{Violation} ($V$), or \textit{No intention to achieve the goal} ($N$) \citep{mira}. This framework provides a more expressive representation of actions, which is useful in applications such as
 task analysis for special education \citep{ta} and robotics \citep{mira_robot}. Building on this foundation, the same authors have also extended the formalism to handle the temporal sequencing of instructions, enabling reasoning about ordered action execution \citep{llmmira}.

\subsection*{Task Planning with Large Language Models}
Of late, LLMs have shown promise in supporting task planning. Although LLMs excel at text generation and general reasoning, they struggle with temporal reasoning. A proposed solution involves translating temporal information into a graph structure, enabling LLMs to learn and reason over it more effectively \citep{xiong}. LLMs do not always produce truthful or consistent responses and may sometimes generate hallucinated or incorrect information. To address these issues and better align model outputs with human expectations, InstructGPT was developed using reinforcement learning from human feedback (RLHF) \citep{llm_human}. However, the sequencing mechanisms of instructions is yet to be addressed.


\subsection*{Positioning of the proposed paper}
The proposed \mimamsa–inspired framework advances this domain by offering a philosophically grounded approach to instruction sequencing across several dimensions, as outlined below:
\begin{itemize}
\item Philosophical Foundation: In contrast to purely algorithmic methods, the proposed framework is rooted in the hermeneutical principles of \mimamsa~philosophy, providing systematic techniques for instruction sequencing.
\item Action–Object Decomposition: By explicitly modeling instructions as action–object pairs, the framework enables more precise dependency analysis than traditional representations based solely on atomic actions.
\item Multiple Sequencing Paradigms: The formalization of three complementary sequencing strategies—direct assertion, purpose-based sequencing, and iterative procedures—offers flexibility for coordinating different instruction types.
\item Formal Verification: The inclusion of deduction rules and completeness proofs ensures formal guarantees of sequence validity, effectively linking philosophical reasoning with computational verification.
\item Cultural Perspective: Drawn from an Indian philosophical perspective, this work expands the western-centric theoretical foundations of reasoning systems.
\end{itemize}

 \section{Extension of Syntax and Semantics for Action-Object Imperative Logic}
 \label{sec:syn_sem}
This section extends the formal syntax and semantics framework of \cite{mira} by refining imperative instructions to explicit action-object pairs.
\subsection{Syntax}
\label{sec:syntax}
The language of imperatives is given by \(\mathcal{L}_i = \langle I, R, P, B \rangle\), where:
\begin{itemize}
  \item \(I = \{ i_1, i_2, \ldots, i_n \}\) is the set of imperatives,
  \item \(R = \{ r_1, r_2, \ldots, r_n \}\) is the set of reasons or Preconditions
  \item \(P = \{ p_1, p_2, \ldots, p_n \}\) is the set of purposes (goals),
  \item \(B = \{ \wedge, \vee, \rightarrow_r, \rightarrow_i, \rightarrow_p \}\) is the set of binary connectives.
\end{itemize}

Here, \(R\) and \(P\) are propositions, following propositional formula syntax. They combine with imperatives \(I\) in several forms to build Imperative Formulas \(\mathcal{F}_i\), specified by Equation \ref{eq:formation}:

\begin{eqnarray}
\label{eq:formation}
\mathcal{F}_i = \{
i \mid i \rightarrow_p p \mid (i \rightarrow_p p_1) \wedge (j \rightarrow_p p_2) \mid (i \rightarrow_p \theta) \oplus (j \rightarrow_p \theta) \mid (\varphi \rightarrow_i \psi) \mid (\tau \rightarrow_r \varphi)
\}
\end{eqnarray}

The fundamental unit of an imperative, denoted \(i\), decomposes into an action and a set of objects. For example, the instruction \emph{``Take a book''} associates the action \emph{``take''} with the object \emph{``book''}.

Formally, this association is a function:
\[
f : I \to A \times \mathcal{P}(O),
\]
where \(I = \{ i_1, i_2, i_3, \ldots, i_n \}\) is a set of instructions,
\(A = \{ a_1, a_2, \ldots, a_k \}\) is a set of actions, and
\(O = \{ o_1, o_2, \ldots, o_m \}\) is a set of objects.

Each instruction \(i_j \in I\) can be represented as:
\begin{eqnarray}
\label{eq:ac_ob}
i_j = (a_j, o_j)
\end{eqnarray}
where \(a_j \in A\), and \(o_j \subseteq O\).

Here, each action can stand alone, or be paired with zero, one, or multiple objects as summarized in Table \ref{tab:ac_ob}.

\begin{table}[h!]
\label{tab:ac_ob}
\centering
\begin{tabular}{|l|l|l|}
\hline
\textbf{Action Type}           & \textbf{Representation}             & \textbf{Example}                \\
\hline
Action with object             & \((a_j, \{ o_k \})\)                & \((\texttt{pick}, \{\texttt{rice}\})\) - \emph{pick rice} \\
\hline
Action with multiple objects   & \((a_j, \{ o_k, o_m \})\)          & \((\texttt{cook}, \{\texttt{rice}, \texttt{pot}\})\) - \emph{cook rice in pot} \\
\hline
Action without object          & \((a_j, \emptyset)\)                & \((\texttt{wait}, \emptyset)\) - \emph{wait} \\
\hline
\end{tabular}
\caption{Instruction representations for actions paired with zero, one, or multiple objects.}
\end{table}


\subsection{Semantics}
The semantics defines how each imperative formula is interpreted over the model, assigning it a satisfaction status based on system states, actions, objects, and goal-directed intentions.
A semantic model \(\mathcal{M}\) is defined as:
\begin{eqnarray}
\label{eq:model}
\mathcal{M} = \langle \mathcal{R}, \mathcal{A}, \mathcal{O}, \mathcal{G}, intention, eval \rangle,
\end{eqnarray}
where
\begin{itemize}
    \item \(\mathcal{R}\) is the set of Preconditions,
    \item \(\mathcal{A}\) is the set of actions,
    \item \(\mathcal{O}\) is the set of objects,
    \item \(\mathcal{G}\) is the set of goals,
    \item \(intention : \mathcal{R} \times \mathcal{G} \to \{ true, false \}\),
    \item \(eval : \mathcal{R} \times (\mathcal{A} \times \mathcal{O}) \to \{ S, V, N \}\).
\end{itemize}

The semantic evaluations for different imperatives are given as follows:

\begin{itemize}
 \item \textbf{Unconditional Imperatives}:
\[
eval(r, (a, o)) =
\begin{cases}
S, & \text{if the action } a \text{ is successfully performed on } o \text{ in } r, \\
V, & \text{otherwise}.
\end{cases}
\]

\item \textbf{Imperative Enjoining Goal}:
\[
eval(r, (a, o) \rightarrow_p g) =
\begin{cases}
S, & \text{if the agent intends } g \text{ and } (a,o) \text{ is satisfied in } r, \\
V, & \text{if the agent intends } g \text{ and } (a,o) \text{ is violated in } r, \\
N, & \text{if there is no intention to achieve goal } g.
\end{cases}
\]

\item \textbf{Imperatives in Sequence}:
\[
eval(r, (a_1, o_1) \rightarrow_i (a_2, o_2)) =
\begin{cases}
S, & \text{if } eval(r, (a_1, o_1)) = S \text{, } eval(r, (a_2, o_2)) = S \text{ and object consistency holds,} \\
V, & \text{otherwise}
\end{cases}
\]

\item \textbf{Imperatives in Parallel}:
\[
eval((r_1, (a_1, o_1)) \wedge (r_2, (a_2, o_2))) =
\begin{cases}
S, & \text{if } eval(r_1, (a_1, o_1)) = S, \quad eval(r_2, (a_2, o_2)) = S, \\
& \text{and object consistency does not hold,} \\
V, & \text{if } eval(r_1, (a_1, o_1)) = V \text{ or } eval(r_2, (a_2, o_2)) = V.
\end{cases}
\]

\item \textbf{Imperatives with Choice}:
\[
eval((r_1, (a_1, o_1)) \oplus (r_2, (a_2, o_2))) =
\begin{cases}
S, & \text{if } eval(r_1, (a_1, o_1)) = S \text{ or } eval(r_2, (a_2, o_2)) = S, \\
V, & \text{otherwise}
\end{cases}
\]
\end{itemize}
This formal action-object representation provides a foundation for capturing the ordering and dependency relationships among instructions. The philosophy of \mimamsa\ includes several principled methods for sequencing such instructions to achieve coherent execution. The next section introduces these methods and formally develops three sequencing strategies by extending this representation.

\section{Sequencing Methods}
\label{sec:sequencing}
According to the Indian philosophical system of \mimamsa, a set of instructions can be sequenced using six distinct ordering principles to ensure coherent and uninterrupted execution. These are: Direct Assertion (\textit{Śrutikrama}), Purpose-Based Sequencing (\textit{Arthakrama}), Order as Given (\textit{P\={a}\d{t}hakrama}), Position-Based Order (\textit{Sth\={a}nakrama}), Principal Activity-Based Order (\textit{Mukhyakrama}), and Iterative Procedure (\textit{Pravri\d{t}\d{t}ikrama}). For more details on the sequencing aspects from the philosophy, the reader may refer to the prior work on temporal ordering of instructions \citep{llmmira}. Among these, three methods—Direct Assertion, Purpose-Based Sequencing, and Iterative Procedure—are formalized in this paper and discussed in Sections \ref{sec:sruti}, \ref{sec:artha}, and \ref{sec:pravritti}, respectively.


\subsection{Direct Assertion (\'{S}rutikrama)}
\label{sec:sruti}
In this type, instructions are provided in a direct and sequential manner. Following the notation from the work of sequencing methods based on \mimamsa~\citep{llmmira}, let:
\begin{itemize}
 \item $i_t = (a_t,o_t)$: instruction at the first instant $t$, with action $a_t$ and object(s) $o_t$
\item $i_{t+1} = (a_{t+1},o_{t+1})$: instruction at the next instant $t+1$
 \end{itemize}
Then, Instruction in sequence can be expressed by Equation \ref{eq:sruti}.
\begin{eqnarray}
\label{eq:sruti}
 (a_t o_t) \rightarrow_i (a_{t+1}o_{t+1})
\end{eqnarray}
where $\rightarrow_i$ denotes temporal sequencing of actions on objects. The indication can be read as  ``perform $a_t$ on $o_t$, then perform $a_{t+1}$ on $o_{t+1}$''.

For a sequence of $n$ instructions, the temporal order and precedence can be formally represented using nested left brackets as shown in Equation \ref{eq:sruti_ext}.
\begin{eqnarray}
 \label{eq:sruti_ext}
(\cdots ((a_1 o_1) \rightarrow_i (a_2 o_2)) \rightarrow_i (a_3 o_3) \cdots ) \rightarrow_i (a_n o_n)
\end{eqnarray}

This representation indicates that each instruction must be completed before the next instruction.

For example, consider three statements \textit{``pick rice''}, \textit{``cook rice in pot''}, \textit{``add rice to dish''}. These can be represented as:
\begin{eqnarray}
 (((pick \{rice\}) \rightarrow_i (cook \{rice,
 pot\})) \rightarrow_i (add \{rice,dish\}))
\end{eqnarray}
Here, objects are progressively updated across instructions. For instance, “rice” becomes “cooked rice” after executing the instruction “cook rice.” This transformation indicates that the instructions are linked through evolving object states, a relationship known as \textbf{object dependency}.

This type of representation serves two major purposes.

\begin{enumerate}
 \item It indicates the temporal order of the actions.
 \item The dependencies of objects the across each step is enforced.
\end{enumerate}
\subsection{Sequencing based on purpose}
\label{sec:artha}
In this type, each instruction is of the form $(\tau \rightarrow_r (i \rightarrow_p p))$ \citep{llmmira}, indicating there is a precondition ($\tau$) for the instruction ($i$) to take place, inorder to achieve the goal ($p$). Here, $\rightarrow_r$ and $\rightarrow_p$ denote ``because of reason (indicating precondition)'' and ``inorder to achieve  goal'', respectively.

This representation can be extended to a series of instructions as given by Equation \ref{eq:artha_ext}.
\begin{eqnarray}
 \label{eq:artha_ext}
 (r_1 \rightarrow_r (i_1 \rightarrow_p p_1)), (r_2 \rightarrow_r (i_2 \rightarrow_p p_2), ..., (r_n \rightarrow_r (i_n \rightarrow_p p_n))
\end{eqnarray}

If the purpose $p_k$ of instruction $i_k$ becomes the precondition $r_{k+1}$ for the next instruction $i_{k+1}$, then $i_k$ precedes $i_{k+1}$, because $r_{k+1} = p_k$. This relation signifies that the second insruction ($i_{k+1}$) depends on the first ($i_k$) and is referred to as \textbf{functional dependency} and has already been used in task analysis for special education \citep{ta}.

Extending this further into the representation  of $i_j$ as ($a_j,o_j$) pair, Equation \ref{eq:artha_ext} can be formalized as shown in Equation \ref{eq:ao_artha}.
\begin{eqnarray}
 \label{eq:ao_artha}
\forall j \in \{1,...,n-1\}: r_j \rightarrow_r ((a_j,o_j) \rightarrow_p p_j), r_{j+1} = p_j
\end{eqnarray}

\subsection{Sequential and Parallel Execution Methods for Repetitive Tasks}
\label{sec:pravritti}
In some tasks, it is necessary to repeat the same action multiple times across different items. Two common approaches to sequence such repetitive actions have been identified in previous work with LLMs \citep{llmmira}, reiterated here for clarity.

Consider the task of a teacher grading answer scripts from 20 students, each answer script containing 5 questions. In the \textbf{Sequential Completion Method}, the teacher grades all five questions of the first student's script before moving on to completely grade the second student’s script, continuing this process sequentially for all students. Alternatively, the \textbf{Step-by-Step Parallel Method}, also known as the \textbf{Iterative Procedure}, involves the teacher grading the first question across all 20 scripts before proceeding to grade the second question for all scripts, and so forth. The formal representation of these two methods are given below.


\subsubsection{Sequential Completion Method}
In this method, the full sequence is performed on one object and the same sequence is repeated for all other objects.

Formally, it can be represented as follows:

Let there be $n$ actions $A = \{a_1,a_2,...,a_n\}$ and $T$ objects for each action, $O_k = \{o_{k1},o_{k2},...,o_{kT}\}$ for $1\leq k \leq n$.

For each object $o_j (1 \leq j \leq T)$, the sequence is given by Equation \ref{eq:seq_single}.
\begin{eqnarray}
 \label{eq:seq_single}
 (a_1o_{1j} \rightarrow_i a_2o_{2j} \rightarrow_i... \rightarrow_i a_no_{nj})
\end{eqnarray}

This is repeated for all $j$ as shown below.


\begin{eqnarray}
\label{eq:seq_m}
\bigseq_{j=1}^T (a_1 o_{1j} \rightarrow_i \cdots \rightarrow_i a_n o_{nj})
\end{eqnarray}


Equation \ref{eq:seq_m} can be interpreted as:
\begin{itemize}
 \item For each object $j$, all actions are performed in sequence before moving to the next object.
 \item The objects involved in sequence are $(o_{1j},o_{2j},...o_{nj})$ for $j$.
\end{itemize}

\subsubsection{Step-by-Step Parallel Method (Iterative Procedure)}
The step-by-step parallel method can be formalized using a parallel composition connective \(\parallel_i\), which groups actions performed independently on different objects without enforcing temporal sequencing or object dependency among them.

Formally, this method is represented as follows:
\begin{eqnarray}
\label{eq:step_parallel_formal}
\Bigl(
(a_1 o_{11} \parallel_i a_1 o_{12} \parallel_i \cdots \parallel_i a_1 o_{1T})
\rightarrow_i \\ \nonumber
(a_2 o_{21} \parallel_i a_2 o_{22} \parallel_i \cdots \parallel_i a_2 o_{2T})
\rightarrow_i \cdots \rightarrow_i \\ \nonumber
(a_n o_{n1} \parallel_i a_n o_{n2} \parallel_i \cdots \parallel_i a_n o_{nT})
\Bigr).
\end{eqnarray}

Here:
\begin{itemize}
  \item \(\parallel_i\) groups the same action \(a_k\) applied to objects \(o_{k1}, \ldots, o_{kT}\) in parallel, without a strict order or dependency between object-specific actions.
  \item \(\rightarrow_i\) imposes a temporal ordering between these groups, requiring all actions in a given group to complete before the next group begins.
\end{itemize}

In this method, the first action is performed on all objects, followed by second action and so on. Same action is grouped and distributed across objects before moving to the next action.

In both the Sequential Completion and Step-by-Step Parallel Methods, object dependency is preserved in accordance with the principle of direct assertion (śrutikrama). Specifically, within the sequential composition $\rightarrow_i$, each instruction group or action sequence enforces the temporal order and dependency relations established by direct assertion, ensuring coherent and consistent execution even when tasks are performed repetitively over multiple objects.

Using these sequencing mechanisms, validity can be logically determined through object dependency and consistency across subsequent instructions, as detailed in the following section.
\section{Object Dependency and Functional Dependency in Instruction Sequencing}
\label{sec:dependency}

The validity of an instruction sequence relies on two main conditions: \textbf{object dependency} and \textbf{functional dependency} across instructions. These ensure that the instruction sequence respects the logical dependencies and state transformations of objects involved.

\subsection{Object Dependency Condition}
\begin{defi}
\textbf{Object Dependency Condition} \\
A sequence of instructions \( I = \{i_1, i_2, \ldots, i_n\} \), where each \( i_j = (a_j, O_j) \), exhibits valid object dependency if for every pair of consecutive instructions \( (i_j, i_{j+1}) \), there exists at least one object \( o^* \) such that:
\[
o^* \in O_j \cap O_{j+1}.
\]
This signifies that the two instructions share at least one object, establishing a valid dependency consistent with direct assertion (śrutikrama).
\end{defi}

\subsection{Functional Dependency Condition}
\begin{defi}
\textbf{Functional Dependency Condition} \\
For instructions \( i_j = (a_j, O_j) \) and \( i_{j+1} = (a_{j+1}, O_{j+1}) \) with shared objects \( o^* \in O_j \cap O_{j+1} \), the sequence maintains functional dependency if:
\[
s_j(o^*) = s_{j+1}^{\mathrm{req}}(o^*).
\]
Here:
\begin{itemize}
  \item \( s_j(o^*) \) is the state of object \( o^* \) immediately after executing instruction \( i_j \),
  \item \( s_{j+1}^{\mathrm{req}}(o^*) \) is the required state of \( o^* \) to start executing \( i_{j+1} \).
\end{itemize}
This relation captures the classical notion of functional dependency, where the post-state of an object after one instruction determines the input state required for the next.
\end{defi}

\subsection{Validity Theorem for Sequencing}
\begin{thm}
\label{th:validity}
\textbf{Validity of Instruction Sequence with Object and Functional Dependencies} \\
Let \( I = \{i_1, i_2, \ldots, i_n\} \) be a sequence of instructions with \( i_j = (a_j, O_j) \). The sequence is valid if:
\[
\forall j \in \{1, \ldots, n-1\}, \quad \text{if} \quad D(i_j, i_{j+1}) = \text{True},
\]
then:
\[
O_j \cap O_{j+1} \neq \emptyset \quad \text{and} \quad
\forall o^* \in O_j \cap O_{j+1}, \quad s_j(o^*) = s_{j+1}^{\mathrm{req}}(o^*),
\]
where \( D(i_j, i_{j+1}) \) expresses a dependency between \( i_j \) and \( i_{j+1} \).
\end{thm}
\begin{proof}
We prove the theorem by induction on the instruction sequence length, focusing on maintenance of functional dependency (state consistency) between consecutive instructions.

\textbf{Base Case:}

Consider the first instruction \( i_1 = (a_1, O_1) \):
\begin{itemize}
    \item Each object \( o \in O_1 \) is initially in state \( s_1^{\mathrm{init}}(o) \).
    \item There are no preceding instructions, so no dependencies are checked.
    \item Instruction \( i_1 \) is executable as long as its required preconditions on object states are satisfied by the initial states.
\end{itemize}

\textbf{Inductive Step:}

Assume that for all instructions up to step \(j\), the following holds: For every pair \((i_k, i_{k+1})\) with \(1 \leq k \leq j-1\),
\[
O_k \cap O_{k+1} \neq \emptyset \quad \text{and} \quad
\forall o^* \in O_k \cap O_{k+1},\ s_k(o^*) = s_{k+1}^{\mathrm{req}}(o^*)
\]
That is, both object dependency and functional dependency are satisfied for all previous pairs in the sequence.


Now consider the next instruction \( i_{j+1} = (a_{j+1}, O_{j+1}) \):
\begin{itemize}
    \item \textbf{Dependency Check:} If \( \exists o^* \in O_j \cap O_{j+1} \), then \( i_{j+1} \) depends on \( i_j \) through object \( o^* \).
    \item \textbf{Functional Dependency Check:} The state of \( o^* \) after executing \( i_j \), denoted \( s_j(o^*) \), must match the required state for \( i_{j+1} \), denoted \( s_{j+1}^{\mathrm{req}}(o^*) \):
    \[
    s_j(o^*) = s_{j+1}^{\mathrm{req}}(o^*).
    \]
\end{itemize}

This ensures that the functional dependency condition is preserved for \( (i_j, i_{j+1}) \), allowing \( i_{j+1} \) to be validly executed. By the principle of induction, the functional dependency holds for all pairs of instructions in the sequence, thereby establishing validity.
\end{proof}

This theorem can be formalized as follows:

\[
\forall j \in \{1, \ldots, n-1\},\
\text{if}\ D(i_j, i_{j+1}) = \text{True}
\implies
\left( O_j \cap O_{j+1} \neq \emptyset\
\ \text{and}\
\forall o^* \in O_j \cap O_{j+1},\ s_j(o^*) = s_{j+1}^{\mathrm{req}}(o^*) \right)
\]

where $D(i_j, i_{j+1})$ indicates a dependency from $i_j$ to $i_{j+1}$.


\subsubsection{Corollary}
A sequence is invalid if there exists a pair \( (i_j, i_{j+1}) \) such that:
\begin{itemize}
  \item \( O_j \cap O_{j+1} = \emptyset \), indicating no common object and thus no dependency.
  \item Or, there exists an object \( o^* \in O_j \cap O_{j+1} \) for which:
    \[
    s_j(o^*) \neq s_{j+1}^{\mathrm{req}}(o^*),
    \]
    violating the functional dependency condition.
\end{itemize}
\subsection{Deduction Rules for Sequencing}

Extending the original deduction rules from MIRA \citep{mira}, this section provides a representation in terms of $<$\textit{action,object}$>$ pair towards the sequencing methods of direct assertion \ref{sec:sruti} and purpose based approach \ref{sec:artha}. Among the deduction rules, only the rule conditional enjoining action $cni$ with $X = \phi$, $Y = \psi$ and $w = i$ is applicable, which reflects the sequencing according to direct assertion method that includes object dependency condition. Additionally, the deduction rule of purpose based sequencing method is also included.


\paragraph{Object-Consistent Sequencing Rule (OCS)}

Assume $i_1$ leads to $i_2$ and there is at least one object common to both instructions. The deduction rule is:

\[
\frac{
    [i_1] \quad \cdots \quad i_2 \qquad O_1 \cap O_2 \neq \emptyset \text{, } O_1 \cap O_{1} \neq \emptyset \quad \text{and} \quad
\forall o^* \in O_1 \cap O_2
}{
    i_1 \rightarrow_i i_2
} \;\texttt{OCS}
\]

\emph{Explanation:}
This rule states that the sequential composition $i_1 \rightarrow_i i_2$ is derivable if $i_1$ leads to $i_2$ and the two instructions share at least one object. This ensures that sequencing is only permitted when there is object overlap, reflecting the principle of object dependency.

\paragraph{Purpose-Linked Sequencing Rule (PLS)}

Let $i_k$ and $i_{k+1}$ be instructions with preconditions $r_k, r_{k+1}$ and purposes $p_k, p_{k+1}$.

\[
\frac{
  \vdash (r_k \rightarrow_r (i_k \rightarrow_p p_k)) \quad
  \vdash (r_{k+1} \rightarrow_r (i_{k+1} \rightarrow_p p_{k+1})) \quad
  \text{Functional dependency:}\; p_k = r_{k+1}
}{
  \vdash (r_k \rightarrow_r (i_k \rightarrow_p p_k)) \rightarrow (r_{k+1} \rightarrow_r (i_{k+1} \rightarrow_p p_{k+1}))
}PLS
\]

This rule derives a sequenced instruction chain when both components are derivable and the purpose of the first instruction provides the precondition for the second (functional dependency). This implements purpose-based ordering in a set of given instructions.

These rules ensure that instruction sequencing is logically justified only when the relevant object and functional dependencies are maintained. They form the basis for the forthcoming proofs of soundness and completeness, which demonstrate the alignment of proof theory and semantic interpretation in this system.

\section{Soundness and Completeness of Deduction Rules}
\label{sec:sound_complete}
In this section, we establish the soundness and completeness of the deduction rules (OCS and PLS) with respect to the semantic model and the Validity Theorem (Theorem \ref{th:validity}).
Soundness ensures that every sequence of instructions derivable within the framework satisfies the semantic validity conditions, which consists of both object and functional dependencies.
Completeness guarantees the converse, that is every sequence that satisfies these semantic validity conditions is derivable within the framework.

We consider sequences of instructions \( I = \{i_1, i_2, \dots, i_n\} \), where each \( i_j = (a_j, O_j) \), and validity is defined per Theorem \ref{th:validity}. The semantic model \( M = \langle \mathcal{R}, A, O, G, \textit{intention}, \textit{eval} \rangle \) interprets these sequences, with satisfaction statuses S (satisfied), V (violated), or N (no intention to achieve goal). Derivability (\( \vdash \)) refers to proofs using OCS and PLS.

\subsection{Soundness}
\begin{thm}
 If a sequence \( I \) is derivable (\( \vdash I \)) using OCS and PLS, then \( I \) is semantically valid in \( M \) (i.e., it satisfies Theorem \ref{th:validity}: for all dependent consecutive pairs \( (i_j, i_{j+1}) \), \( O_j \cap O_{j+1} \neq \emptyset \) and \( \forall o^* \in O_j \cap O_{j+1} \), \( r_j(o^*) = r_{j+1}^{\text{req}}(o^*) \), and \( \textit{eval}(r, I) = S \) for some initial precondition \( r \in \mathcal{R} \)).
\end{thm}
\begin{proof} We proceed by induction on the length of the derivation.

\textbf{Base Case:} For a single instruction \( i_1 = (a_1, O_1) \), derivability is trivial (no rules applied). Semantic validity holds vacuously (no pairs). By the semantics, \( \textit{eval}(r, (a_1, O_1)) = S \) if \( a_1 \) is performed on objects in \( O_1 \) under precondition \( r \), assuming initial preconditions match requirements.

\textbf{Inductive Step:} Assume soundness holds for derivations of length up to \( k \). Consider a derivation of length \( k+1 \). Since validity requires both object and functional dependencies, derivations involve applications of both OCS and PLS for sequences with such dependencies. The last rule applied could be OCS or PLS, but the full derivation combines them to capture both conditions.

- \textbf{Case OCS (Object Dependency):} The rule derives \( i_1 \rightarrow_i i_2 \) from premises \( [i_1] \dots i_2 \), with \( O_1 \cap O_2 \neq \emptyset \) and object consistency. By inductive hypothesis, the premises are semantically valid (each sub-sequence evaluates to S). Semantics for imperatives in sequence (Section 2.2) yield \( \textit{eval}(r, (a_1, o_1) \rightarrow_i (a_2, o_2)) = S \) if both sub-evaluations are S and object consistency holds. This satisfies the object dependency part of Theorem 1 for the pair. Functional dependency, if required, is preserved through concurrent or prior applications of PLS in the derivation.

- \textbf{Case PLS (Functional Dependency):} The rule derives \( (r_k \rightarrow_r (i_k \rightarrow_p p_k)) \rightarrow (r_{k+1} \rightarrow_r (i_{k+1} \rightarrow_p p_{k+1})) \) from premises with functional dependency \( p_k = r_{k+1} \). By inductive hypothesis, premises are valid. Semantics for purpose-enjoined imperatives ensure \( \textit{eval}(r, (a_k, o_k) \rightarrow_p g) = S \) only if intention holds and the action satisfies the goal. Specifically, in the context of functional dependency: (i) \( (r_k \rightarrow_r (i_k \rightarrow_p p_k)) \) evaluates to S if the intention of the goal \( p_k \) is true and the action \( i_k \) is performed with precondition \( r_k \) true; (ii) after execution, the intention of goal \( p_k \) becomes reality (i.e., the evaluation from step (i) turns to a true value, serving as the precondition for the next step \( r_{k+1} \)); (iii) the next \( (r_{k+1} \rightarrow_r (i_{k+1} \rightarrow_p p_{k+1})) \) then becomes S when the intention of goal \( p_{k+1} \) is true and action \( i_{k+1} \) is performed with precondition \( r_{k+1} \). The linkage \( p_k = r_{k+1} \) enforces functional dependency yielding overall S. Object dependency, if required, is enforced through concurrent or prior OCS applications in the full derivation (e.g., ensuring shared objects with consistent preconditions), satisfying both parts of Theorem \ref{th:validity}.

Derivations combine OCS and PLS as needed (e.g., OCS for object overlap and PLS for purpose-precondition chaining in mixed sequences), ensuring both dependencies are captured and leading to overall semantic validity by induction.
\end{proof}
\subsection{Completeness}
\begin{thm}
 If a sequence \( I \) is semantically valid in \( M \) (i.e., satisfies Theorem \ref{th:validity} and \( \textit{eval}(r, I) = S \) for some \( r \in \mathcal{R} \)), then \( I \) is derivable (\( \vdash I \)) using OCS and PLS.
 \end{thm}

 \begin{proof}
Again, by induction on the sequence length \( n \).

\textbf{Base Case:} For \( n=1 \), \( i_1 \) is valid if \( \textit{eval}(r, (a_1, O_1)) = S \). No rules needed; derivability is trivial.

\textbf{Inductive Step:} Assume completeness for sequences up to length \( n-1 \). For length \( n \), since \( I \) is valid, all consecutive pairs satisfy both object and functional dependencies per Theorem 1.

The sub-sequence \( \{i_1, \dots, i_{n-1}\} \) is valid by assumption, so derivable by inductive hypothesis. To extend to the full sequence, apply both OCS and PLS as required by the dependencies:

- Apply OCS for object dependency (pairs share objects \( O_j \cap O_{j+1} \neq \emptyset \) and preconditions match), deriving temporal sequencing \( i_{n-1} \rightarrow_i i_n \).

- Apply PLS for functional dependency (pairs link via \( p_j = r_{j+1} \)), deriving purpose-chained sequencing, where the semantic evaluation to S follows the steps: (i) intention of \( p_{n-1} \) true and action performed under \( r_{n-1} \); (ii) post-execution realization of \( p_{n-1} \) as true, becoming \( r_n \); (iii) subsequent evaluation to S for the next imperative under \( r_n \) with intention of \( p_n \) true.

Since Theorem \ref{th:validity} requires both dependencies for validity, apply both rules for each relevant pair: OCS to enforce object overlap and precondition consistency, and PLS to enforce purpose-precondition linkage. For mixed sequences, decompose into sub-sequences, derive them using the appropriate rule(s), then combine. Semantic validity ensures both conditions align with the premises of OCS and PLS, so the full derivation exists by combining the rules.

By induction, all semantically valid sequences are derivable.

These proofs align the proof theory (deduction rules) with the semantics, ensuring the framework's logical consistency for instruction sequencing.
\end{proof}


\section{Conclusion and Future Work}
\label{sec:conclude}

In this paper, a logical formalism for sequencing instructions has been introduced. The fundamental building block of this framework consists of an action-object pair, extracted from an individual instruction. Three sequencing approaches, namely Direct Assertion, known as Śrutikrama, Purpose-Driven Sequencing, referred to as Arthakrama, and Iterative Parallel Sequencing, designated as Pravṛttikrama, have been incorporated, drawing inspiration from the Indian philosophical tradition of \mimamsa. Furthermore, the validity theorem, supported by soundness and completeness proofs for the deduction rules labeled as OCS and PLS, has been developed to guarantee the logical accuracy of the entire framework.

The proposed framework is computationally instantiated through the MIRA AI Agent, a system leveraging Large Language Models (e.g., Groq, Gemini) for instruction generation and sequence validation. This agent, detailed in future work, demonstrates the practical feasibility of our semantic model, with real-time deployment as a web application. Current efforts focus on formal verification, with implementation specifics to be elaborated in a forthcoming paper.



\bibliographystyle{apalike} % or any style you prefer, e.g., unsrtnat for numerical order
\bibliography{mira_ref}

\end{document}
